\documentclass[12pt,a4paper,]{report}
\usepackage{lmodern}
\usepackage{setspace}
\setstretch{1.15}
\usepackage{amssymb,amsmath}
\usepackage{ifxetex,ifluatex}
\usepackage{fixltx2e} % provides \textsubscript
\ifnum 0\ifxetex 1\fi\ifluatex 1\fi=0 % if pdftex
  \usepackage[T1]{fontenc}
  \usepackage[utf8]{inputenc}
\else % if luatex or xelatex
  \ifxetex
    \usepackage{mathspec}
  \else
    \usepackage{fontspec}
  \fi
  \defaultfontfeatures{Ligatures=TeX,Scale=MatchLowercase}
    \setmainfont[]{Liberation Serif}
\fi
% use upquote if available, for straight quotes in verbatim environments
\IfFileExists{upquote.sty}{\usepackage{upquote}}{}
% use microtype if available
\IfFileExists{microtype.sty}{%
\usepackage{microtype}
\UseMicrotypeSet[protrusion]{basicmath} % disable protrusion for tt fonts
}{}
\usepackage[margin=1in]{geometry}
\usepackage{hyperref}
\hypersetup{unicode=true,
            pdftitle={Ανάλυση γενετικής ποικιλότητας και γονιδιακής ροής καστανεώνα και απομονωμένου φυσικού πληθυσμού καστανιάς στη νήσο Λέσβο με ουδέτερους και μη μοριακούς δείκτες},
            pdfauthor={Νικόλαος Τουρβάς},
            pdfborder={0 0 0},
            breaklinks=true}
\urlstyle{same}  % don't use monospace font for urls
\usepackage{longtable,booktabs}
\usepackage{graphicx,grffile}
\makeatletter
\def\maxwidth{\ifdim\Gin@nat@width>\linewidth\linewidth\else\Gin@nat@width\fi}
\def\maxheight{\ifdim\Gin@nat@height>\textheight\textheight\else\Gin@nat@height\fi}
\makeatother
% Scale images if necessary, so that they will not overflow the page
% margins by default, and it is still possible to overwrite the defaults
% using explicit options in \includegraphics[width, height, ...]{}
\setkeys{Gin}{width=\maxwidth,height=\maxheight,keepaspectratio}
\IfFileExists{parskip.sty}{%
\usepackage{parskip}
}{% else
\setlength{\parindent}{0pt}
\setlength{\parskip}{6pt plus 2pt minus 1pt}
}
\setlength{\emergencystretch}{3em}  % prevent overfull lines
\providecommand{\tightlist}{%
  \setlength{\itemsep}{0pt}\setlength{\parskip}{0pt}}
\setcounter{secnumdepth}{0}
% Redefines (sub)paragraphs to behave more like sections
\ifx\paragraph\undefined\else
\let\oldparagraph\paragraph
\renewcommand{\paragraph}[1]{\oldparagraph{#1}\mbox{}}
\fi
\ifx\subparagraph\undefined\else
\let\oldsubparagraph\subparagraph
\renewcommand{\subparagraph}[1]{\oldsubparagraph{#1}\mbox{}}
\fi

%%% Use protect on footnotes to avoid problems with footnotes in titles
\let\rmarkdownfootnote\footnote%
\def\footnote{\protect\rmarkdownfootnote}

%%% Change title format to be more compact
\usepackage{titling}

% Create subtitle command for use in maketitle
\newcommand{\subtitle}[1]{
  \posttitle{
    \begin{center}\large#1\end{center}
    }
}

\setlength{\droptitle}{-2em}

  \title{Ανάλυση γενετικής ποικιλότητας και γονιδιακής ροής καστανεώνα και
απομονωμένου φυσικού πληθυσμού καστανιάς στη νήσο Λέσβο με ουδέτερους
και μη μοριακούς δείκτες}
    \pretitle{\vspace{\droptitle}\centering\huge}
  \posttitle{\par}
  \subtitle{~}
  \author{Νικόλαος Τουρβάς}
    \preauthor{\centering\large\emph}
  \postauthor{\par}
      \predate{\centering\large\emph}
  \postdate{\par}
    \date{Οκτώβριος 2018}

\usepackage[greek]{babel}

\begin{document}
\maketitle

\section{Εισαγωγή}

Γιατί πρέπει να μελετηθεί η γεν. ποικ - διαφ

Σημασία γενετικής ποικ Long term see introduction\textsuperscript{1}
Selection may erode genetic diversity, which is a crucial factor for the
success of breeding programs.

Γεωργία vs δασοπονία φυσικοί πληθυσμοί και Γανό σελ
19\textsuperscript{2} τι γίνεται με τους καστανεώνες, δες Πετρόπουλος.
βοήθεια σε προγρ βελτίωσης: Γανό σελ 22

Σημαντικό είδος γιατί\ldots{}

Καρπός ξύλο γενετική βελτίωση σε χώρες - όπως Μαλλιαρού για οξυά

\hypertarget{-}{%
\section{Θεωρητικό πλαίσιο}\label{-}}

Refugia - Gene flow - Long term\textsuperscript{3,4} Founder effect από
ανθρω

Δυο λόγια για τη διεθνή βιβλιογραφία. Υπάρχει ή δεν υπάρχει ποικιλότητα;
"Γενικα'έχει βρεθεί ότι \ldots{} {[}\ldots{}{]} Συγκεκριμένα στην
Ελλάδα\ldots{}

\hypertarget{-}{%
\section{Ερευνητικά ερωτήματα}\label{-}}

\begin{itemize}
\tightlist
\item
  γενετικός χαρακτηρισμός καστανεώνα και κατ' επέκταση της ποικιλίας και
  φυσικού πληθυσμού
\item
  μελέτη γονιδιακής ροής μεταξύ των δύο πληθυσμών
\end{itemize}

\section{Μεθοδολογία}

\hypertarget{----dna}{%
\subsection{Συλλογή φυτικού υλικού \& Εκχύλιση DNA}\label{----dna}}

Γενετικό υλικό παρέχεται από το Ινστιτούτο Γενετικής Βελτίωσης\\
CTAB

\hypertarget{---}{%
\subsection{Αλυσιδωτή αντίδραση πολυμεράσης \& Γενοτύπηση}\label{---}}

10 SSR citation??? θα χρησιμοποιήσουμε SSR γιατί είναι κατάλληλοι για τη
μελέτη πρόσφατων μικροεξελικτικών διαδικασιών\textsuperscript{5} Πώς θα
γίνουν οι ομάδες multiplex --\textgreater{} diveRsity ABI3730xl
σκοράρισμα --\textgreater{} GeneMapper

\hypertarget{---}{%
\subsection{Βασικές παράμετροι γενετικής ποικιλότητας}\label{---}}

ΜΑΛΛΟΝ ΠΡΕΠΕΙ ΝΑ ΦΤΙΑΞΩ ΠΙΝΑΚΑ ΜΕ ΛΟΓΙΣΜΙΚΑ ΚΑΙ ΤΙ ΘΑ ΚΑΝΟΥΝ
R\textsuperscript{6} adegenet\textsuperscript{7}
poppr\textsuperscript{8} hierfstat\textsuperscript{9}
genepop\textsuperscript{10} pegas ape fangorn

αριθμός διαφορετικών γενοτύπων (multilocus genotypes) Allelic richness -
indicator να σημειωθεί ML\_Null-Freq LGP Fst outliers --\textgreater{}
BAYESCAN outflank

NeEstimator -\textgreater{} BOTTLENECK

R Notebook - R markdown

\subsection{διαφοροποίηση}

Nei's Ds, Cav-Sf για φυλογενετικά\textsuperscript{11}
F\textsubscript{ST}, G''\textsubscript{ST}, D\textsubscript{est} PCA, CA
AMOVA φυλογενετική ape STRUCTURE DAPC

\hypertarget{-}{%
\subsection{Γενετική τοπίου}\label{-}}

γονιδιακή ροη - Nm=1/4FSt και MIGRATE IBD - mantel test - adegenet, ape
Genetic boundaries γενετικά φράγματα φραγμοί- Monmonier's algorithm sPCA
Geneland DIYABC ??? GeneClass2 ???

\hypertarget{-}{%
\section{Αναμενόμενα αποτελέσματα}\label{-}}

Τι θα μάθουμε από αυτή τη μελέτη\ldots{} χαρακτηρισμός ποικιλίας Λέσβου
gene flow

\hypertarget{-}{%
\section{Χρηματοδότηση / Χρονοδιάγραμμα}\label{-}}

\textsuperscript{12}

\section{Βιβλιογραφία}

\hypertarget{refs}{}
\leavevmode\hypertarget{ref-pollegioni_long-term_2011}{}%
1. Pollegioni, P. \emph{et al.} Long-term human impacts on genetic
structure of Italian walnut inferred by SSR markers. \emph{Tree Genetics
\& Genomes} \textbf{7,} 707--723 (2011).

\leavevmode\hypertarget{ref-ganopoulos__2013}{}%
2. Ganopoulos, I. Διερεύνηση γενετικής ποικιλότητας, ταυτοποίηση και
εφαρμογή λειτουργικών δεικτών στην κερασιά (\emph{Prunus avium} L.).
(2013).

\leavevmode\hypertarget{ref-petit_comparative_2005}{}%
3. Petit, R. J. \emph{et al.} Comparative organization of chloroplast,
mitochondrial and nuclear diversity in plant populations.
\emph{Molecular Ecology} \textbf{14,} 689--701 (2005).

\leavevmode\hypertarget{ref-petit_glacial_2003}{}%
4. Petit, R. J. \emph{et al.} Glacial refugia: Hotspots but not melting
pots of genetic diversity. \emph{Science} \textbf{300,} 1563--1565
(2003).

\leavevmode\hypertarget{ref-wang_recognizing_2010}{}%
5. Wang, I. J. Recognizing the temporal distinctions between landscape
genetics and phylogeography. \emph{Molecular Ecology} \textbf{19,}
2605--2608 (2010).

\leavevmode\hypertarget{ref-r_core_team_r:_2018}{}%
6. R Core Team. R: A Language and Environment for Statistical Computing.
(2018).

\leavevmode\hypertarget{ref-jombart_adegenet:_2008}{}%
7. Jombart, T. Adegenet: A R package for the multivariate analysis of
genetic markers. \emph{Bioinformatics} \textbf{24,} 1403--1405 (2008).

\leavevmode\hypertarget{ref-kamvar_poppr_2014}{}%
8. Kamvar, Z. N., Tabima, J. F. \& Grünwald, N. J. Poppr : An R package
for genetic analysis of populations with clonal, partially clonal,
and/or sexual reproduction. \emph{PeerJ} \textbf{2,} e281 (2014).

\leavevmode\hypertarget{ref-goudet_hierfstat_2004}{}%
9. Goudet, J. Hierfstat, a package for R to compute and test
hierarchical F-statistics. (2004).

\leavevmode\hypertarget{ref-rousset_genepop007:_2008}{}%
10. Rousset, F. Genepop'007: A complete re-implementation of the genepop
software for Windows and Linux. \emph{Molecular Ecology Resources}
\textbf{8,} 103--106 (2008).

\leavevmode\hypertarget{ref-takezaki_genetic_1996}{}%
11. Takezaki, N. \& Nei, M. Genetic distances and the setting of
conservation priorities. \emph{Biological Conservation} \textbf{75,} 311
(1996).

\leavevmode\hypertarget{ref-villani_genetic_1991}{}%
12. Villani, F., Pigliucci, M., Benedettelli, S. \& Cherubini, M.
Genetic differentiation among Turkish chestnut (\emph{Castanea sativa}
Mill.) Populations. \emph{Heredity} \textbf{66,} 131--136 (1991).


\end{document}
