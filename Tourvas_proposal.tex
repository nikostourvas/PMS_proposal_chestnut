\documentclass[12pt,a4paper,]{report}
\usepackage{lmodern}
\usepackage{setspace}
\setstretch{1.15}
\usepackage{amssymb,amsmath}
\usepackage{ifxetex,ifluatex}
\usepackage{fixltx2e} % provides \textsubscript
\ifnum 0\ifxetex 1\fi\ifluatex 1\fi=0 % if pdftex
  \usepackage[T1]{fontenc}
  \usepackage[utf8]{inputenc}
\else % if luatex or xelatex
  \ifxetex
    \usepackage{mathspec}
  \else
    \usepackage{fontspec}
  \fi
  \defaultfontfeatures{Ligatures=TeX,Scale=MatchLowercase}
    \setmainfont[]{Liberation Serif}
\fi
% use upquote if available, for straight quotes in verbatim environments
\IfFileExists{upquote.sty}{\usepackage{upquote}}{}
% use microtype if available
\IfFileExists{microtype.sty}{%
\usepackage{microtype}
\UseMicrotypeSet[protrusion]{basicmath} % disable protrusion for tt fonts
}{}
\usepackage[margin=1in]{geometry}
\usepackage{hyperref}
\hypersetup{unicode=true,
            pdftitle={Ανάλυση γενετικής ποικιλότητας και γονιδιακής ροής καστανεώνα και απομονωμένου φυσικού πληθυσμού καστανιάς στη νήσο Λέσβο με ουδέτερους και μη μοριακούς δείκτες},
            pdfauthor={Νικόλαος Τουρβάς},
            pdfborder={0 0 0},
            breaklinks=true}
\urlstyle{same}  % don't use monospace font for urls
\usepackage{longtable,booktabs}
\usepackage{graphicx,grffile}
\makeatletter
\def\maxwidth{\ifdim\Gin@nat@width>\linewidth\linewidth\else\Gin@nat@width\fi}
\def\maxheight{\ifdim\Gin@nat@height>\textheight\textheight\else\Gin@nat@height\fi}
\makeatother
% Scale images if necessary, so that they will not overflow the page
% margins by default, and it is still possible to overwrite the defaults
% using explicit options in \includegraphics[width, height, ...]{}
\setkeys{Gin}{width=\maxwidth,height=\maxheight,keepaspectratio}
\IfFileExists{parskip.sty}{%
\usepackage{parskip}
}{% else
\setlength{\parindent}{0pt}
\setlength{\parskip}{6pt plus 2pt minus 1pt}
}
\setlength{\emergencystretch}{3em}  % prevent overfull lines
\providecommand{\tightlist}{%
  \setlength{\itemsep}{0pt}\setlength{\parskip}{0pt}}
\setcounter{secnumdepth}{0}
% Redefines (sub)paragraphs to behave more like sections
\ifx\paragraph\undefined\else
\let\oldparagraph\paragraph
\renewcommand{\paragraph}[1]{\oldparagraph{#1}\mbox{}}
\fi
\ifx\subparagraph\undefined\else
\let\oldsubparagraph\subparagraph
\renewcommand{\subparagraph}[1]{\oldsubparagraph{#1}\mbox{}}
\fi

%%% Use protect on footnotes to avoid problems with footnotes in titles
\let\rmarkdownfootnote\footnote%
\def\footnote{\protect\rmarkdownfootnote}

%%% Change title format to be more compact
\usepackage{titling}

% Create subtitle command for use in maketitle
\newcommand{\subtitle}[1]{
  \posttitle{
    \begin{center}\large#1\end{center}
    }
}

\setlength{\droptitle}{-2em}

  \title{Ανάλυση γενετικής ποικιλότητας και γονιδιακής ροής καστανεώνα και
απομονωμένου φυσικού πληθυσμού καστανιάς στη νήσο Λέσβο με ουδέτερους
και μη μοριακούς δείκτες}
    \pretitle{\vspace{\droptitle}\centering\huge}
  \posttitle{\par}
  \subtitle{~}
  \author{Νικόλαος Τουρβάς}
    \preauthor{\centering\large\emph}
  \postauthor{\par}
      \predate{\centering\large\emph}
  \postdate{\par}
    \date{Οκτώβριος 2018}

\usepackage[greek]{babel}

\begin{document}
\maketitle

\section{Εισαγωγή}

Η ευρωπαϊκή καστανιά (\emph{Castanea sativa} Mill.) είναι ένα από τα πιο
διαδεδομένα και οικονομικά σημαντικά είδη της λεκάνης της Μεσογείου.
Εκμεταλλεύεται τόσο για τους καρπούς της όσο και για την ξυλεία της και
για αυτό το λόγο καλλιεργείται σε περιοχές πλέον της μεσογειακής ζώνης
(Fernández-López and Alia, 2003). Η εκτεταμένη καλλιέργεια και μεταφορά
φυτογενετικού υλικού καστανιάς από τον άνθρωπο ήδη από την αρχαιότητα,
έχει αφήσει το στίγμα της στην εξελικτική πορεία του είδους. Στη
γεωργία, οι περισσότερες καλλιέργειες, έχουν απωλέσει το μεγαλύτερο
μέρος των γενετικών πόρων που διέθεταν πριν την βελτιστοποίηση της
παραγωγικής διαδικασίας μέσω της επιλογής άριστων κλώνων. Αποτέλεσμα
αυτής της διαδικασίας είναι οι εγκαθιδρυμένες φυτείες να έχουν μικρά
περιθώρια προσαρμογής σε περιβαλλοντικές διαταραχές, αλλά και βελτίωσης
ως προς νέα γνωρίσματα (Reed and Frankham, 2003).

Αυτή η τάση παρατηρείται σε διάφορες διαβαθμίσεις και στους καστανεώνες
της χώρας μας οι οποίοι μπορούν να διακριθούν σε δύο κατηγορίες:
εντατικώς καλλιεργούμενοι (εμπορικώς εκμεταλλευόμενα εμβόλια σε
επιλεγμένα υποκείμενα εντός αγροτικών γαιών) και εκτατικώς
καλλιεργούμενοι (εμβόλια από επιλεγμένο υλικό τοπικής προέλευσης σε
υποκείμενα άγριου τύπου εντός των δασών). Όσον αφορά τους πρώτους έχει
βρεθεί ότι τυπικά φέρουν υψηλό αριθμό διαφορετικών γενοτύπων
(\textasciitilde{}80\% του συνολικού αριθμού δέντρων), ενώ οι δεύτεροι
έχουν σαφώς χαμηλότερα ποσοστά (Aravanopoulos and Drouzas, 2005; Martin
et al., 2009). Υψηλοί αριθμοί διαφορετικών κλώνων θέτουν προβλήματα
ομοιομορφίας της παραγωγής, αλλά αναμένεται να έχουν καλύτερη απόκριση
στις διαφαινόμενες συνθήκες της κλιματικής αλλαγής (Aravanopoulos et
al., 2010). Ένα πρόβλημα που είναι δυνατόν να προκύψει από την
καλλιέργεια καστανεώνων σε μια περιοχή είναι η πιθανότητα εισαγωγής
ξένου γενετικού υλικού σε φυσικούς πληθυσμούς μέσω της ροής γονιδίων. Το
υλικό αυτό δύναται να αυξήσει το γενετικό φορτίο των φυσικών συστάδων
μειώνοντας έτσι την προσαρμοστικότητά τους (Lefèvre, 2004; Olden et al.,
2004).

Η ικανότητα των δασών καστανιάς να αναπαράγονται βλαστικά έχει επιτρέψει
στον άνθρωπο να διαχειρίζεται αυτά τα οικοσυστήματα με μικρούς
περίτροπους χρόνους. Ωστόσο στα πρεμνοφυή δάση παρεμποδίζεται η φυσική
αναγέννηση άρα και η δράση της φυσικής επιλογής σε αυτό το εξελικτικό
στάδιο. Επιπλέον τα πρεμνοφυή δάση αναμένεται να έχουν χαμηλότερο
δραστικό μέγεθος πληθυσμού συγκρινόμενα με αντίστοιχα φυσικά δάση
καστανιάς που αναπαράγονται εγγενώς (Mattioni et al., 2008). Τα παραπάνω
στοιχεία συνηγορούν στο γεγονός ότι τα δάση αυτά έχουν εν δυνάμει
μικρότερο δυναμικό προσαρμογής ή/και παρουσιάζουν χρονική υστέρηση στην
προσαρμογή σε περιπτώσεις ταχείας περιβαλλοντικής μεταβολής
(Aravanopoulos and Drouzas, 2003).

Μελέτες που έχουν πραγματοποιηθεί με αλληλοένζυμα (Villani et al.,
1991a, 1991b, 1994) έδειξαν χαμηλότερες τιμές ποικιλότητας σε
ευρωπαϊκούς πληθυσμούς σε Ιταλία και Γαλλία σε σχέση με πληθυσμούς
προερχόμενους από την Τουρκία. Οι ελληνικοί πληθυσμοί του είδους έχουν
διερευνηθεί σε μελέτες (Aravanopoulos and Drouzas, 2005; Aravanopoulos
et al., 2005, 2010, 2001) κυρίως με τη χρήση αλληλοενζύμων. Τα τελευταία
χρόνια εμφανίζεται πληθώρα εργασιών στη διεθνή βιβλιογραφία με αναλύσεις
σύγχρονων μοριακών δεικτών όπως οι μικροδορυφόροι (SSR) (Buck et al.,
2003; Marinoni et al., 2003; Martin et al., 2010; Mattioni et al., 2008,
2013).

Σήμερα κρίνεται αναγκαία η μελέτη των ελληνικών πληθυσμών του είδους με
σύγχρονες μοριακές τεχνικές. Υψηλή προτεραιότητα θα πρέπει να έχουν
μεταξύ άλλων και γεωγραφικά απομονωμένοι πληθυσμοί καθώς ο κίνδυνος
απώλειας πολύτιμου γενετικού υλικού από αυτούς, λόγω φαινομένων όπως η
γενετική εκτροπή είναι υψηλός. Προτείνεται η μελέτη δύο πληθυσμών
καστανιάς στην περιοχή Αγιάσου της νήσου Λέσβου. Ο ένας πληθυσμός
εντοπίζεται σε φυσική συστάδα η οποία διαχειρίζεται πρεμνοφυώς
(βλαστικός ανα-πολλαπλασιασμός), ενώ ο δεύτερος αντιπροσωπεύει εκτατικώς
καλλιεργούμενο καστανεώνα ο οποίος έχει ιδρυθεί με τη χρήση εμβολίων σε
άγριου-τύπου υποκείμενα.

\section{Σκοπός}

Σκοποί της παρούσας πρότασης είναι:

\begin{enumerate}
\def\labelenumi{\arabic{enumi}.}
\item
  Η μελέτη της γενετικής ποικιλότητας και διαφοροποίησης δύο πληθυσμών
  καστανιάς (καστανεώνας - φυσικός πληθυσμός) στη νήσο Λέσβο με τη χρήση
  μοριακών δεικτών μικροδορυφόρων SSR.
\item
  Η μελέτη γονιδιακής ροής μεταξύ των δύο πληθυσμών.
\item
  Η καταγραφή των κλώνων (multilocus genotypes) που συγκροτούν τον
  καστανεώνα και κατ' επέκταση την ποικιλία κάστανων Αγιάσου στη νήσο
  Λέσβο.
\end{enumerate}

\section{Μεθοδολογία}

\hypertarget{----dna}{%
\subsubsection{\texorpdfstring{\emph{Συλλογή φυτικού υλικού \& Εκχύλιση
DNA}}{Συλλογή φυτικού υλικού \& Εκχύλιση DNA}}\label{----dna}}

Θα διεξαχθεί δειγματοληψία κατ' ελάχιστον 25-30 ατόμων ανά πληθυσμό.
Όσον αφορά των πληθυσμό του καστανεώνα, επειδή ενδέχεται να εντοπίζονται
πολλαπλά εμβόλια σε ορισμένα υποκείμενα (Aravanopoulos et al., 2010), θα
ληφθεί φυτικός ιστός από πολλαπλούς κλάδους ``ύποπτων'' δέντρων. Η
εκχύλιση του γενωμικού DNA θα πραγματοποιηθεί με τη μέθοδο CTAB (Doyle
and Doyle, 1987).

\hypertarget{--}{%
\subsubsection{\texorpdfstring{\emph{Αλυσιδωτή αντίδραση
πολυμεράσης}}{Αλυσιδωτή αντίδραση πολυμεράσης}}\label{--}}

Για τη γενετική ανάλυση θα χρησιμοποιηθούν μοριακοί δείκτες SSR (Simple
Sequence Repeats). Οι SSR ή μικροδορυφορικοί δείκτες αποτελούνται από
αλληλουχίες 2-5 νουκλεοτιδίων που επαναλαμβάνονται 5-100 φορές (Budak et
al., 2004). Ο συγκυρίαρχος χαρακτήρας τους σε συνδυασμό με τα υψηλά
επίπεδα πολυμορφισμού που παρουσιάζουν, τους καθιστούν εξαιρετικούς
δείκτες για τη μελέτη πρόσφατων μικροεξελικτικών διεργασιών (Wang,
2010). Οι μικροδορυφόροι μπορούν να διακριθούν σε ουδέτερους και
λειτουργικούς. Στην πρώτη κατηγορία ανήκουν εκείνες οι γονιδιακές θέσεις
οι οποίες βρίσκονται σε τυχαία τμήματα του γονιδιώματος. Αντίθετα οι
λειτουργικοί δείκτες ενισχύουν γονιδιακές θέσεις που εκφράζονται και
επομένως σχετίζονται με κάποια βιολογική διεργασία (Durand et al.,
2010).

Συγκεκριμένα θα χρησιμοποιηθούν τρεις ουδέτεροι δείκτες SSR (EMCs38,
CsCAT3, CsCAT6) (Buck et al., 2003; Marinoni et al., 2003) και πέντε
δείκτες EST-SSR (GOT004, GOT021, FIR110, POR042 και WAG004) (Durand et
al., 2010). Οι EST-SSR που έχουν επιλεγεί σχετίζονται με την αντοχή των
δέντρων σε συνθήκες υδατικής έλλειψης.

\hypertarget{---}{%
\subsubsection{\texorpdfstring{\emph{Γενετική ποικιλότητα εντός
πληθυσμών}}{Γενετική ποικιλότητα εντός πληθυσμών}}\label{---}}

Η στατιστική ανάλυση θα πραγματοποιηθεί σε περιβάλλον της στατιστικής
γλώσσας R (R Core Team, 2018) με χρήση των πακέτων poppr (Kamvar et al.,
2014), hierfstat (Goudet, 2005), και genepop (Rousset, 2008). Οι βασικές
παράμετροι γενετικής ποικιλότητας (αριθμός αλληλομόρφων, ετεροζυγωτία,
συντελεστής ομομειξίας, ισορροπία Hardy-Weinberg κ.α.) θα εκτιμηθούν
αφού πρώτα διερευνηθεί η ανισορροπία σύνδεσης μεταξύ των γονιδιακών
θέσεων και η πιθανότητα ύπαρξης μηδενικών αλληλομόρφων.

Τα δραστικά μεγέθη των πληθυσμών θα εκτιμηθούν με το πρόγραμμα
N\textsubscript{e} Estimator (Do et al., 2014), ενώ η ύπαρξη γενετικών
στενωπών στην εξελικτική ιστορία των πληθυσμών θα διερευνηθεί με τα
προγράμματα BOTTLENECK (Piry et al., 1999) (πρόσφατο παρελθόν: 2
N\textsubscript{e} - 4 N\textsubscript{e} γενεές) και Arlequin
(Excoffier and Lischer, 2010) (M-ratio test, υποδεικνύει παλαιότερες
δημογραφικές μεταβολές).

Επίσης θα υπολογιστεί ο αριθμός των γενοτύπων (multilocus genotypes) που
απαντώνται στον υπό μελέτη καστανεώνα.

\hypertarget{---}{%
\subsubsection{\texorpdfstring{\emph{Γενετική ποικιλότητα μεταξύ
πληθυσμών}}{Γενετική ποικιλότητα μεταξύ πληθυσμών}}\label{---}}

Θα υπολογιστεί ο συντελεστής διαφοροποίησης F\textsubscript{ST} (Nei,
1987), η γενετική απόσταση κατά Nei (1978), καθώς και η γεωμετρική
απόσταση Cavalli-Sforza (Cavalli-Sforza and Edwards, 1967) με το πακέτο
hierfstat της R (Goudet, 2005). Οι πολυμεταβλητές αναλύσεις PCA, CA και
η ανάλυση μοριακής διακύμανσης AMOVA θα διεξαχθούν με το πακέτο adegenet
(Jombart, 2008). Το λογισμικό Structure (Pritchard et al., 2000), αλλά
και η πολυμεταβλητή ανάλυση DAPC (Jombart et al., 2010) (σε περιβάλλον
R) θα χρησιμοποιηθoύν για την περαιτέρω διερεύνηση της διαφοροποίησης
των δύο πληθυσμών. H DAPC ενδέχεται να αποδειχθεί ιδιαίτερα χρήσιμη εάν
οι παραδοχές που απαιτεί το Structure δεν ικανοποιούνται (π.χ.
ανισορροπία Hardy-Weinberg).

\hypertarget{-}{%
\subsubsection{\texorpdfstring{\emph{Γονιδιακή
ροή}}{Γονιδιακή ροή}}\label{-}}

Ο αριθμός των μεταναστών θα εκτιμηθεί αρχικά σύμφωνα με το κλασσικό
μοντέλο \(Nm = [(1 / F_{ST}) - 1] / 4\) (Frankham et al., 2002). Ωστόσο,
αποτελεί παραδοχή της μεθόδου ότι η μετανάστευση είναι τάξεις μεγεθους
μεγαλύτερη σε σχέση με τις μεταλλάξεις. Λόγω του υψηλού ρυθμού
μεταλλάξεων που χαρακτηρίζει τους μικροδορυφόρους (περίπου
\(5 * 10^{-4}\)) (Whittaker et al., 2003) αυτή η παραδοχή ενδέχεται να
παραβιάζεται (Whitlock, 2011). Για το λόγο αυτό, ο αριθμός των
μεταναστών θα υπολογιστεί και με τη χρήση των εξειδικευμένων λογισμικών
(coalescent-based models) Migrate (Beerli and Felsenstein, 2001) και
MIGRAINE (De Iorio et al., 2005).

\hypertarget{-}{%
\section{Αναμενόμενα αποτελέσματα}\label{-}}

H καταγραφή της γενετικής ποικιλότητας των πληθυσμών αναμένεται να
συνδράμει στο έργο της διαφύλαξης και διαχείρισης των γενετικών πόρων
της καστανιάς στη χώρα μας.

Επιπροσθέτως, η μελέτη της γονιδιακής ροής θα αποσαφηνίσει την έκταση
ενός ενδεχομένου φαινομένου γενετικής ρύπανσης που υφίσταται ο φυσικός
πληθυσμός από τον καστανεώνα. Η γνώση αυτή έχει άμεση πρακτική αξία όσον
αφορά τις ενέργειες που πρέπει να ληφθούν για την διατήρηση της
γενετικής ταυτότητας των φυσικών πληθυσμών της περιοχής.

Ο γενετικός χαρακτηρισμός της ποικιλίας Αγιάσου Λέσβου θα επιτρέψει την
ανάπτυξη ελέγχων ταυτοποίησης και ιχνηλασιμότητας για το συγκεκριμένο
φυτογενετικό υλικό και θα οδηγήσει μεσοπρόθεσμα στην καλύτερη
εκμετάλλευση/εμπορική χρήση της ποικιλίας.

\section{Βιβλιογραφία}

\hypertarget{refs}{}
\leavevmode\hypertarget{ref-aravanopoulos_does_2003}{}%
Aravanopoulos, F.A., and Drouzas, A.D. (2003). Does forest management
influence genetic diversity in chestnut (\emph{Castanea} \emph{sativa}
Mill.) Populations? In 11th Pan-Hellenic Forest Science Conference,
(Olympia, Greece), pp. 329--337.

\leavevmode\hypertarget{ref-aravanopoulos_multilocus_2005}{}%
Aravanopoulos, F., and Drouzas, A. (2005). Multilocus genetic structure
of European chestnut (\emph{Castanea} \emph{sativa}) Hellenic clones and
genetic diversity of orchard population. Acta Horticulturae 447--452.

\leavevmode\hypertarget{ref-aravanopoulos_molecular_2005}{}%
Aravanopoulos, F., Bucci, G., Akkak, A., Blanco Silva, R., Botta, R.,
Buck, E., Cherubini, M., Drouzas, A., Fernández-López, J., Mattioni, C.,
et al. (2005). Molecular population genetics and dynamics of chestnut
(\emph{Castanea} \emph{sativa}) in Europe: Inferences for gene
conservation and tree improvement. Acta Horticulturae 403--412.

\leavevmode\hypertarget{ref-aravanopoulos_genetic_2010}{}%
Aravanopoulos, F., Mitsakaki, A., Katsidi, E., Avramidou, E., and
Sfakianaki, E. (2010). Genetic diversity of intensively and extensively
managed chestnut (\emph{Castanea} \emph{sativa}) orchards in Greece.
Acta Horticulturae 121--126.

\leavevmode\hypertarget{ref-Aravanopoulos2001}{}%
Aravanopoulos, F.A., Drouzas, A.D., and Alizoti, P.G. (2001).
Electrophoretic and quantitative variation in chestnut (\emph{Castanea}
\emph{sativa} Mill.) In Hellenic populations in old-growth natural and
coppice stands. Forest, Snow and Landscape Research \emph{76}, 429--434.

\leavevmode\hypertarget{ref-beerli_maximum_2001}{}%
Beerli, P., and Felsenstein, J. (2001). Maximum likelihood estimation of
a migration matrix and effective population sizes in n subpopulations by
using a coalescent approach. PNAS \emph{98}, 4563--4568.

\leavevmode\hypertarget{ref-buck_isolation_2003}{}%
Buck, E.J., Hadonou, M., James, C.J., Blakesley, D., and Russell, K.
(2003). Isolation and characterization of polymorphic microsatellites in
European chestnut (\emph{Castanea} \emph{sativa} Mill.). Molecular
Ecology Notes \emph{3}, 239--241.

\leavevmode\hypertarget{ref-budak_comparative_2004}{}%
Budak, H., Shearman, R.C., Parmaksiz, I., and Dweikat, I. (2004).
Comparative analysis of seeded and vegetative biotype buffalograsses
based on phylogenetic relationship using ISSRs, SSRs, RAPDs, and SRAPs.
Theor Appl Genet \emph{109}, 280--288.

\leavevmode\hypertarget{ref-cavalli1967phylogenetic}{}%
Cavalli-Sforza, L.L., and Edwards, A.W. (1967). Phylogenetic analysis:
Models and estimation procedures. Evolution \emph{21}, 550--570.

\leavevmode\hypertarget{ref-de_iorio_stepwise_2005}{}%
De Iorio, M., Griffiths, R.C., Leblois, R., and Rousset, F. (2005).
Stepwise mutation likelihood computation by sequential importance
sampling in subdivided population models. Theoretical Population Biology
\emph{68}, 41--53.

\leavevmode\hypertarget{ref-Do2014}{}%
Do, C., Waples, R.S., Peel, D., Macbeth, G.M., Tillett, B.J., and
Ovenden, J.R. (2014). NeEstimator v2: Re-implementation of software for
the estimation of contemporary effective population size (Ne) from
genetic data. Molecular Ecology Resources \emph{14}, 209--214.

\leavevmode\hypertarget{ref-doyle1987genomic}{}%
Doyle, J., and Doyle, J. (1987). Genomic plant DNA preparation from
fresh tissue-CTAB method. Phytochem Bull \emph{19}, 11--15.

\leavevmode\hypertarget{ref-durand_fast_2010}{}%
Durand, J., Bodénès, C., Chancerel, E., Frigerio, J.-M., Vendramin, G.,
Sebastiani, F., Buonamici, A., Gailing, O., Koelewijn, H.-P., Villani,
F., et al. (2010). A fast and cost-effective approach to develop and map
EST-SSR markers: Oak as a case study. BMC Genomics \emph{11}, 570.

\leavevmode\hypertarget{ref-excoffier_arlequin_2010}{}%
Excoffier, L., and Lischer, H.E.L. (2010). Arlequin suite ver 3.5: A new
series of programs to perform population genetics analyses under Linux
and Windows. Molecular Ecology Resources \emph{10}, 564--567.

\leavevmode\hypertarget{ref-fernandez-lopez_euforgen_2003}{}%
Fernández-López, J., and Alia, R. (2003). EUFORGEN Technical Guidelines
for genetic conservation and use for chestnut (\emph{Castanea}
\emph{sativa}) (Rome, Italy: Internation Plant Genetic Resources
Institute).

\leavevmode\hypertarget{ref-frankham_introduction_2002}{}%
Frankham, R., Briscoe, D.A., and Ballou, J.D. (2002). Introduction to
conservation genetics (Cambridge university press).

\leavevmode\hypertarget{ref-goudet_hierfstat_2005}{}%
Goudet, J. (2005). Hierfstat, a package for r to compute and test
hierarchical F-statistics. Molecular Ecology Notes \emph{5}, 184--186.

\leavevmode\hypertarget{ref-Jombart2008}{}%
Jombart, T. (2008). Adegenet: A R package for the multivariate analysis
of genetic markers. Bioinformatics \emph{24}, 1403--1405.

\leavevmode\hypertarget{ref-Jombart2010}{}%
Jombart, T., Devillard, S., and Balloux, F. (2010). Discriminant
analysis of principal components: A new method for the analysis of
genetically structured populations. BMC Genetics \emph{11}, 94.

\leavevmode\hypertarget{ref-Kamvar2014}{}%
Kamvar, Z.N., Tabima, J.F., and Grünwald, N.J. (2014). Poppr : An R
package for genetic analysis of populations with clonal, partially
clonal, and/or sexual reproduction. PeerJ \emph{2}, e281.

\leavevmode\hypertarget{ref-Lefevre2004}{}%
Lefèvre, F. (2004). Human impacts on forest genetic resources in the
temperate zone : An updated review. \emph{197}, 257--271.

\leavevmode\hypertarget{ref-marinoni_development_2003}{}%
Marinoni, D., Akkak, A., Bounous, G., Edwards, K.J., and Botta, R.
(2003). Development and characterization of microsatellite markers in
Castanea sativa (Mill.). Molecular Breeding \emph{11}, 127--136.

\leavevmode\hypertarget{ref-martin_identification_2009}{}%
Martin, M.A., Alvarez, J.B., Mattioni, C., Cherubini, M., Villani, F.,
and Martin, L.M. (2009). Identification and characterisation of
traditional chestnut varieties of southern Spain using morphological and
simple sequence repeat (SSRs) markers. Annals of Applied Biology
\emph{154}, 389--398.

\leavevmode\hypertarget{ref-martin_genetic_2010}{}%
Martin, M.A., Mattioni, C., Cherubini, M., Taurchini, D., and Villani,
F. (2010). Genetic diversity in European chestnut populations by means
of genomic and genic microsatellite markers. Tree Genetics \& Genomes
\emph{6}, 735--744.

\leavevmode\hypertarget{ref-Mattioni2008}{}%
Mattioni, C., Cherubini, M., Micheli, E., Villani, F., and Bucci, G.
(2008). Role of domestication in shaping \emph{Castanea} \emph{sativa}
genetic variation in Europe. Tree Genetics \& Genomes \emph{4},
563--574.

\leavevmode\hypertarget{ref-Mattioni2013}{}%
Mattioni, C., Martin, M.A., Pollegioni, P., Cherubini, M., and Villani,
F. (2013). Microsatellite markers reveal a strong geographical structure
in European populations of \emph{Castanea} \emph{sativa} (Fagaceae):
Evidence for multiple glacial refugia. American Journal of Botany
\emph{100}, 951--961.

\leavevmode\hypertarget{ref-Nei1978}{}%
Nei, M. (1978). Estimation of average heterozygosity and genetic
distance from a small number of individuals. Genetics \emph{89},
583--590.

\leavevmode\hypertarget{ref-nei1987molecular}{}%
Nei, M. (1987). Molecular evolutionary genetics (Columbia university
press).

\leavevmode\hypertarget{ref-olden_ecological_2004}{}%
Olden, J.D., LeRoy Poff, N., Douglas, M.R., Douglas, M.E., and Fausch,
K.D. (2004). Ecological and evolutionary consequences of biotic
homogenization. Trends in Ecology \& Evolution \emph{19}, 18--24.

\leavevmode\hypertarget{ref-Piry1999}{}%
Piry, S., Luikart, G., and Cornuet, J.-M. (1999). BOTTLENECK: A Computer
Program for Detecting Recent Reductions in the Effective Population Size
Using Allele Frequency Data. The Journal of Heredity Ann Hum Genet
\emph{90}, 253--259.

\leavevmode\hypertarget{ref-pritchard_inference_2000}{}%
Pritchard, J.K., Stephens, M., and Donnelly, P. (2000). Inference of
Population Structure Using Multilocus Genotype Data. Genetics
\emph{155}, 945--959.

\leavevmode\hypertarget{ref-RCoreTeam2018}{}%
R Core Team (2018). R: A Language and Environment for Statistical
Computing (R Foundation for Statistical Computing).

\leavevmode\hypertarget{ref-Reed2003}{}%
Reed, D.H., and Frankham, R. (2003). Correlation between Fitness and
Genetic Diversity. Conservation Biology \emph{17}, 230--237.

\leavevmode\hypertarget{ref-Rousset2008}{}%
Rousset, F. (2008). Genepop'007: A complete re-implementation of the
genepop software for Windows and Linux. Molecular Ecology Resources
\emph{8}, 103--106.

\leavevmode\hypertarget{ref-Villani1991}{}%
Villani, F., Pigliucci, M., Benedettelli, S., and Cherubini, M. (1991a).
Genetic differentiation among Turkish chestnut (\emph{Castanea}
\emph{sativa} Mill.) Populations. Heredity \emph{66}, 131--136.

\leavevmode\hypertarget{ref-villani1991genetic}{}%
Villani, F., Benedettelli, S., Paciucci, M., Cherubini, M., and
Pigliucci, M. (1991b). Genetic variation and differentiation between
natural populations of chestnut (\emph{Castanea} \emph{sativa} Mill.)
From Italy. Biochemical Markers in the Population Genetics of Forest
Trees, Eds. Fineschi, ME Malvolti, F. Cannata, HH Hattemer)--SPB Acad.
Publishing, the Hague, Netherlands 91--103.

\leavevmode\hypertarget{ref-villani1994evolution}{}%
Villani, F., Pigliucci, M., and Cherubini, M. (1994). Evolution of
\emph{Castanea} \emph{sativa} Mill, in Turkey and Europe. Genetics
Research \emph{63}, 109--116.

\leavevmode\hypertarget{ref-Wang2010a}{}%
Wang, I.J. (2010). Recognizing the temporal distinctions between
landscape genetics and phylogeography. Molecular Ecology \emph{19},
2605--2608.

\leavevmode\hypertarget{ref-whitlock_gst_2011}{}%
Whitlock, M.C. (2011). G'ST and D do not replace FST. Molecular Ecology
\emph{20}, 1083--1091.

\leavevmode\hypertarget{ref-whittaker_likelihood-based_2003}{}%
Whittaker, J.C., Harbord, R.M., Boxall, N., Mackay, I., Dawson, G., and
Sibly, R.M. (2003). Likelihood-Based Estimation of Microsatellite
Mutation Rates. Genetics \emph{164}, 781--787.


\end{document}
