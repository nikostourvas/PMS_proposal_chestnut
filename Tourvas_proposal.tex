\documentclass[12pt,a4paper,]{report}
\usepackage{lmodern}
\usepackage{setspace}
\setstretch{1.15}
\usepackage{amssymb,amsmath}
\usepackage{ifxetex,ifluatex}
\usepackage{fixltx2e} % provides \textsubscript
\ifnum 0\ifxetex 1\fi\ifluatex 1\fi=0 % if pdftex
  \usepackage[T1]{fontenc}
  \usepackage[utf8]{inputenc}
\else % if luatex or xelatex
  \ifxetex
    \usepackage{mathspec}
  \else
    \usepackage{fontspec}
  \fi
  \defaultfontfeatures{Ligatures=TeX,Scale=MatchLowercase}
    \setmainfont[]{Liberation Serif}
\fi
% use upquote if available, for straight quotes in verbatim environments
\IfFileExists{upquote.sty}{\usepackage{upquote}}{}
% use microtype if available
\IfFileExists{microtype.sty}{%
\usepackage{microtype}
\UseMicrotypeSet[protrusion]{basicmath} % disable protrusion for tt fonts
}{}
\usepackage[margin=1in]{geometry}
\usepackage{hyperref}
\hypersetup{unicode=true,
            pdftitle={Ανάλυση γενετικής ποικιλότητας και γονιδιακής ροής καστανεώνα και απομονωμένου φυσικού πληθυσμού καστανιάς στη νήσο Λέσβο με ουδέτερους και μη μοριακούς δείκτες},
            pdfauthor={Νικόλαος Τουρβάς},
            pdfborder={0 0 0},
            breaklinks=true}
\urlstyle{same}  % don't use monospace font for urls
\usepackage{longtable,booktabs}
\usepackage{graphicx,grffile}
\makeatletter
\def\maxwidth{\ifdim\Gin@nat@width>\linewidth\linewidth\else\Gin@nat@width\fi}
\def\maxheight{\ifdim\Gin@nat@height>\textheight\textheight\else\Gin@nat@height\fi}
\makeatother
% Scale images if necessary, so that they will not overflow the page
% margins by default, and it is still possible to overwrite the defaults
% using explicit options in \includegraphics[width, height, ...]{}
\setkeys{Gin}{width=\maxwidth,height=\maxheight,keepaspectratio}
\IfFileExists{parskip.sty}{%
\usepackage{parskip}
}{% else
\setlength{\parindent}{0pt}
\setlength{\parskip}{6pt plus 2pt minus 1pt}
}
\setlength{\emergencystretch}{3em}  % prevent overfull lines
\providecommand{\tightlist}{%
  \setlength{\itemsep}{0pt}\setlength{\parskip}{0pt}}
\setcounter{secnumdepth}{0}
% Redefines (sub)paragraphs to behave more like sections
\ifx\paragraph\undefined\else
\let\oldparagraph\paragraph
\renewcommand{\paragraph}[1]{\oldparagraph{#1}\mbox{}}
\fi
\ifx\subparagraph\undefined\else
\let\oldsubparagraph\subparagraph
\renewcommand{\subparagraph}[1]{\oldsubparagraph{#1}\mbox{}}
\fi

%%% Use protect on footnotes to avoid problems with footnotes in titles
\let\rmarkdownfootnote\footnote%
\def\footnote{\protect\rmarkdownfootnote}

%%% Change title format to be more compact
\usepackage{titling}

% Create subtitle command for use in maketitle
\newcommand{\subtitle}[1]{
  \posttitle{
    \begin{center}\large#1\end{center}
    }
}

\setlength{\droptitle}{-2em}

  \title{Ανάλυση γενετικής ποικιλότητας και γονιδιακής ροής καστανεώνα και
απομονωμένου φυσικού πληθυσμού καστανιάς στη νήσο Λέσβο με ουδέτερους
και μη μοριακούς δείκτες}
    \pretitle{\vspace{\droptitle}\centering\huge}
  \posttitle{\par}
  \subtitle{~}
  \author{Νικόλαος Τουρβάς}
    \preauthor{\centering\large\emph}
  \postauthor{\par}
      \predate{\centering\large\emph}
  \postdate{\par}
    \date{Οκτώβριος 2018}

\usepackage[greek]{babel}

\begin{document}
\maketitle

\section{Εισαγωγή}

Η ευρωπαϊκή καστανιά (\emph{Castanea sativa} Mill.) είναι ένα από τα πιο
διαδεδομένα και οικονομικά σημαντικά είδη της λεκάνης της Μεσογείου
(cite euforgen). Εκμεταλλεύεται τόσο για τους καρπούς της όσο και για
την ξυλεία της και για αυτό το λόγο καλλιεργείται σε περιοχές πλέον της
μεσογειακής ζώνης (citation?).

Η μεγάλη εξάρτηση κοινωνιών έχει οδηγήσει\ldots{} Εξαιτίας της
εκτεταμένης ανθρωπογενούς μεταφοράς γενετικού υλικού\ldots{}, γενετικό
υλικό του είδους έχει τύχει εκτεταμένης μεταφοράς (δες στα παπερ για
ancient και κάνε cite) εδώ και αιώνες. Επιπλέον η χρήση επιλεγμένων
γενοτύπων σε καστανεώνες αλλά και πρεμνοφυής διαχείριση των φύσικών
πληθυσμών έχουν επιφέρει σημαντικές απώλειες γενετικών πόρων του είδους
(αραβανοπ? - παπαδ). Η πίεση αυτή έχει ισχυροποιηθεί - ενταθεί λόγω
προσβολών κυρίως από τους μύκητες \emph{Cryphonectria parasitica} και
\emph{Phytophthora cambivora}. + κατακερματισμός απομόνωση γενετικού
υλικού

ΔΕΣ Παπαδήμα Καστοριά

\hypertarget{-}{%
\section{Θεωρητικό πλαίσιο}\label{-}}

\begin{enumerate}
\def\labelenumi{\arabic{enumi}.}
\item
  Σημασία γενετικής ποικ Long term see introduction (Pollegioni et al.,
  2011) γανο σελ 20-22 Selection may erode genetic diversity, which is a
  crucial factor for the success of breeding programs. Γεωργία vs
  δασοπονία φυσικοί πληθυσμοί και Γανό σελ 19 ({\textbf{???}}) τι
  γίνεται με τους καστανεώνες, δες Πετρόπουλος.
  (\(\Pi\)\(\epsilon\)\(\tau\)\(\rho\)ό\(\pi\)ο\(\upsilon\)\(\lambda\)ος,
  2016)

  τι γινεται με τα coppice --\textgreater{} (Aravanopoulos, 2018)
\end{enumerate}

Refugia - Gene flow - Long term (Petit et al., 2003, 2005) Founder
effect από ανθρω

(κίνδυνος για γονιδιακή ροή - μόλυνση στον φυσικό πληθυσμό)

Δυο λόγια για τη διεθνή βιβλιογραφία. Υπάρχει ή δεν υπάρχει ποικιλότητα;
"Γενικα'έχει βρεθεί ότι \ldots{} {[}\ldots{}{]} Συγκεκριμένα στην
Ελλάδα\ldots{}

\begin{enumerate}
\def\labelenumi{\arabic{enumi}.}
\tightlist
\item
  Η λεπτομερής γνώση της δομής και της σύνθεσης της γενετικής
  παραλλακτικότητας αποτελεί απαραίτητη προϋπόθεση για την προστασία και
  διαχείριση των γενετικών πόρων ενός είδους. Στη γεωργία, οι
  περισσότερες καλλιέργειες, έχουν απολέσει το μεγαλύτερο μέρος των
  γενετικών πόρων που διέθεταν πριν την βελτιστοποίηση της παραγωγικής
  διαδικασίας μέσω της επιλογής ποικιλιών και άριστων κλώνων.
  \ldots{}kastanewnes\ldots{} Αυτό έχει σαν αποτέλεσμα να δυσχερένεται
  το έργο των βελτιωτών στην εύρεση κατάλληλων γενοτύπων για
  διασταυρώσεις και να μειώνεται το δυναμικό προσαρμογής ως προς νέα
  γνωρίσματα.
\end{enumerate}

Αυτό το φαινόμενο παρατηρείται στους καστανεώνες της χώρας μας οι οποίοι
συνήθως δημιουργούνται με εμβόλια επιλεγμένων κλώνων σε υποκείμενα
άγριου τύπου. coppice - πρεμνοφυής διαχείριση - τα πρεμνοφυή δάση
αναπαράγονται βλαστικά και αρα αναμένεται να έχουν χαμηλότερο δραστικό
μέγεθος πληθυσμού συγκρινόμενα με όμοιας έκτασης φυσικά δάση καστανιάς
που αναπαράγονται εγγενώς (Mattioni et al., 2008).{[}hill{]} Το
χαμηλότερο δραστικό μέγεθος πληθυσμού υποδεικνύει ότι το γενετικό
απόθεμα από το οποίο προκύπτει η εξέλιξη του δάσους είναι μικρό και άρα
ενδυνάμει μικρότερο δυναμικό προσαρμογής ή/και πιο αργό βαθμό
εξελικτικής προσαρμογής σε περιπτώσεις ταχείας περιβαλλοντικής μεταβολής
{[}47{]}.

refugia - founder effects

γονιδιακή ροή

\hypertarget{-}{%
\section{Ερευνητικά ερωτήματα}\label{-}}

Προτείνεται η μελέτη δύο πληθυσμών καστανιάς της νήσου Λέσβου. Ο ένας
πληθυσμός εντοπίζεται σε φυσική συστάδα η οποία διαχειρίζεται πρεμνοφυώς
(βλαστικός πολλαπλασιασμός), ενώ ο δεύτερος αντιπροσωπεύει καστανεώνα ο
οποίος έχει ιδρυθεί με τη χρήση εμβολίων σε άγριου-τύπου υποκείμενα.

Σκοποί της παρούσας πρότασης είναι:

\begin{enumerate}
\def\labelenumi{\arabic{enumi}.}
\item
  Η μελέτη της γενετικής ποικιλότητας και διαφοροποίησης δύο πληθυσμών
  καστανιάς (καστανεώνας - φυσικός πληθυσμός) στη νήσο Λέσβο με τη χρήση
  μοριακών δεικτών μικροδορφόρων SSR.
\item
  Ο γενετικός χαρακτηρισμός των πολυγονιδιακών γενοτύπων (multilocus
  genotypes) που συγκροτούν τον καστανεώνα και κατ' επέκταση την
  ποικιλίας καστανιάς στη νήσο Λέσβο.
\item
  Η μελέτη γονιδιακής ροής μεταξύ των δύο πληθυσμών.
\end{enumerate}

\section{Μεθοδολογία}

\hypertarget{----dna}{%
\subsection{Συλλογή φυτικού υλικού \& Εκχύλιση DNA}\label{----dna}}

Θα διεξαχθεί δειγματοληψία κατ' ελάχιστον 25-30 ατόμων ανά πληθυσμό. Η
εκχύλιση του γενωμικού DNA θα πραγματοποιηθεί με τη μέθοδο CTAB {[}Doyle
and Doyle 1987{]}

\hypertarget{---}{%
\subsection{Αλυσιδωτή αντίδραση πολυμεράσης \& Γενοτύπηση}\label{---}}

Για τη γενετική ανάλυση θα χρησιμοποιηθούν μοριακοί δείκτες SSR (Simple
Sequence Repeats). Οι SSR ή μικροδορυφορικοί δείκτες αποτελούνται από
αλληλουχίες 2-5 νουκλεοτιδίων που επαναλαμβάνονται 5-100 φορές
({\textbf{???}}). Ο συγκυρίαρχος χαρακτήρας τους σε συνδυασμό με τα
υψηλά επίπεδα πολυμορφισμού που παρουσιάζουν, τους καθιστούν
εξαιρετικούς δείκτες για τη μελέτη πρόσφατων μικροεξελικτικών διεργασιών
(Wang, 2010). Οι μικροδορυφόροι μπορούν να διακριθούν σε ουδέτερους και
λειτουργικούς. Στην πρώτη κατηγορία ανήκουν εκείνες οι γονιδιακές θέσεις
οι οποίες βρίσκονται σε τυχαία τμήματα του γονιδιώματος. Αντίθετα οι
λειτουργικοί δείκτες ενισχύουν γονιδιακές θέσεις που εκφράζονται και
επομένως σχετίζονται με κάποια βιολογική διεργασία {[}citation???{]}.

Συγκεκριμένα θα χρησιμοποιηθούν τρεις ουδέτεροι δείκτες SSR και πέντε
δείκτες EST-SSR. ({\textbf{???}}; {\textbf{???}}; Sullivan et al.,
2013).

λειτουργικοί δείκτες ξηρασία {[}πετροπουλος σελ39{]}

ABI3730xl σκοράρισμα --\textgreater{} GeneMapper

πινακάκι - ποιοι είναι οι δείκτες που χρησιμοποιούνται?

\hypertarget{---}{%
\subsection{Βασικές παράμετροι γενετικής ποικιλότητας}\label{---}}

R (R Core Team, 2018) adegenet (Jombart, 2008) poppr (Kamvar et al.,
2014) hierfstat ({\textbf{???}}) genepop (Rousset, 2008) pegas ape
fangorn

αριθμός διαφορετικών γενοτύπων (multilocus genotypes) Allelic richness -
indicator να σημειωθεί ML\_Null-Freq LGP Fst outliers --\textgreater{}
BAYESCAN outflank

NeEstimator -\textgreater{} M ratio -\textgreater{} υποδεικνύει μείωση
μεγέθους πληθυσμού σε μεγάλα χρονικά διαστήματα (Aleksic and Geburek,
2014) BOTTLENECK -\textgreater{} πρόσφατες γενετικές στενωπούς (2
N\textsubscript{e} - 4 N\textsubscript{e} γενεές)

R Notebook - R markdown

\subsection{διαφοροποίηση}

Nei's Ds, Cav-Sf για φυλογενετικά F\textsubscript{ST},
G''\textsubscript{ST}, D\textsubscript{est} PCA, CA AMOVA φυλογενετική
ape Cavalli-Sforza - δε χρειάζεται για 2 πληθυσμούς. Αν όμως προστεθούν
κ άλλοι STRUCTURE DAPC

\hypertarget{-}{%
\subsection{Γονιδιακή ροή}\label{-}}

δες οπωσδήποτε (Aleksic and Geburek, 2014; Leonarduzzi et al., 2016)
Cervus (see SM1 Leonarduzzi) -\textgreater{} ποσοστό παιδιών που οι
γονείς τους είναι άτομα του πληθυσμού Nm=1/4FSt παραδοσιακή μέθοδος
STRUCTURE μοντέλο πρόδρομης γνώσης πληθυσμών -\textgreater{} δες
({\textbf{???}}; Aleksic and Geburek, 2014) MIGRATE ???

With an estimated average mutation rate of roughly l =5 · 10)4
(Goldstein \& Schlotterer 1999; Whittaker et al.~2003)

In many cases gene flow can be safely assumed to be high relative to
mutation rate, so in much of the literature GST is used to assess
migration rates {[}Whitlock 2011{]}

\hypertarget{-}{%
\subsection{Γενετική τοπίου}\label{-}}

IBD - mantel test - adegenet, ape Genetic boundaries γενετικά φράγματα
φραγμοί- Monmonier's algorithm sPCA Geneland DIYABC ??? GeneClass2 ???

\hypertarget{-}{%
\section{Αναμενόμενα αποτελέσματα}\label{-}}

Τι θα μάθουμε από αυτή τη \#\#\# Γενετική τοπίου IBD - mantel test -
adegenet, ape Genetic boundaries γενετικά φράγματα φραγμοί- Monmonier's
algorithm sPCA Geneland DIYABC ??? GeneClass2 ???μελέτη\ldots{}
χαρακτηρισμός ποικιλίας Λέσβου gene flow

\hypertarget{-}{%
\section{Χρηματοδότηση / Χρονοδιάγραμμα}\label{-}}

(Villani et al., 1991)

\section{Βιβλιογραφία}

\hypertarget{refs}{}
\leavevmode\hypertarget{ref-Aleksic2014}{}%
Aleksic, J.M., Geburek, T., 2014. Quaternary population dynamics of an
endemic conifer, Picea omorika, and their conservation implications.
Conservation Genetics 15, 87--107.
\url{https://doi.org/10.1007/s10592-013-0523-6}

\leavevmode\hypertarget{ref-Aravanopoulos2018a}{}%
Aravanopoulos, F.A., 2018. Do Silviculture and Forest Management Affect
the Genetic Diversity and Structure of Long-Impacted Forest Tree
Populations? Forests 9, 14.

\leavevmode\hypertarget{ref-Jombart2008}{}%
Jombart, T., 2008. adegenet: a R package for the multivariate analysis
of genetic markers. Bioinformatics 24, 1403--1405.
\url{https://doi.org/10.1093/bioinformatics/btn129}

\leavevmode\hypertarget{ref-Kamvar2014}{}%
Kamvar, Z.N., Tabima, J.F., Grünwald, N.J., 2014. Poppr : an R package
for genetic analysis of populations with clonal, partially clonal,
and/or sexual reproduction. PeerJ 2, e281.
\url{https://doi.org/10.7717/peerj.281}

\leavevmode\hypertarget{ref-Leonarduzzi2016a}{}%
Leonarduzzi, C., Piotti, A., Spanu, I., Vendramin, G.G., 2016. Effective
gene flow in a historically fragmented area at the southern edge of
silver fir (Abies alba Mill.) distribution. Tree Genetics \& Genomes 12,
95. \url{https://doi.org/10.1007/s11295-016-1053-4}

\leavevmode\hypertarget{ref-Mattioni2008}{}%
Mattioni, C., Cherubini, M., Micheli, E., Villani, F., Bucci, G., 2008.
Role of domestication in shaping Castanea sativa genetic variation in
Europe. Tree Genetics \& Genomes 4, 563--574.
\url{https://doi.org/10.1007/s11295-008-0132-6}

\leavevmode\hypertarget{ref-Petit2003}{}%
Petit, R.J., Aguinagalde, I., De Beaulieu, J.L., Bittkau, C., Brewer,
S., Cheddadi, R., Ennos, R., Fineschi, S., Grivet, D., Lascoux, M.,
Mohanty, A., Müller-Starck, G., Demesure-Musch, B., Palmé, A., Martín,
J.P., Rendell, S., Vendramin, G.G., 2003. Glacial refugia: Hotspots but
not melting pots of genetic diversity. Science 300, 1563--1565.
\url{https://doi.org/10.1126/science.1083264}

\leavevmode\hypertarget{ref-Petit2005}{}%
Petit, R.J., Duminil, J., Fineschi, S., Hampe, A., Salvini, D.,
Vendramin, G.G., 2005. Comparative organization of chloroplast,
mitochondrial and nuclear diversity in plant populations. Molecular
Ecology 14, 689--701.
\url{https://doi.org/10.1111/j.1365-294X.2004.02410.x}

\leavevmode\hypertarget{ref-Pollegioni2011}{}%
Pollegioni, P., Woeste, K., Olimpieri, I., Marandola, D., Cannata, F.,
Emilia Malvolti, M., 2011. Long-term human impacts on genetic structure
of Italian walnut inferred by SSR markers. Tree Genetics \& Genomes 7,
707--723. \url{https://doi.org/10.1007/s11295-011-0368-4}

\leavevmode\hypertarget{ref-RCoreTeam2018}{}%
R Core Team, 2018. R: A Language and Environment for Statistical
Computing. R Foundation for Statistical Computing, Vienna, Austria.

\leavevmode\hypertarget{ref-Rousset2008}{}%
Rousset, F., 2008. genepop'007: a complete re-implementation of the
genepop software for Windows and Linux. Molecular Ecology Resources 8,
103--106. \url{https://doi.org/10.1111/j.1471-8286.2007.01931.x}

\leavevmode\hypertarget{ref-Sullivan2013}{}%
Sullivan, A.R., Lind, J.F., McCleary, T.S., Romero-Severson, J.,
Gailing, O., 2013. Development and Characterization of Genomic and
Gene-Based Microsatellite Markers in North American Red Oak Species.
Plant Molecular Biology Reporter 31, 231--239.
\url{https://doi.org/10.1007/s11105-012-0495-6}

\leavevmode\hypertarget{ref-Villani1991}{}%
Villani, F., Pigliucci, M., Benedettelli, S., Cherubini, M., 1991.
Genetic differentiation among Turkish chestnut
(\textless{}i\textgreater{}Castanea sativa\textless{}/i\textgreater{}
Mill.) populations. Heredity 66, 131--136.
\url{https://doi.org/10.1038/hdy.1991.16}

\leavevmode\hypertarget{ref-Wang2010a}{}%
Wang, I.J., 2010. Recognizing the temporal distinctions between
landscape genetics and phylogeography. Molecular Ecology 19, 2605--2608.
\url{https://doi.org/10.1111/j.1365-294X.2010.04715.x}

\leavevmode\hypertarget{ref-2016a}{}%
\(\Pi\)\(\epsilon\)\(\tau\)\(\rho\)ό\(\pi\)ο\(\upsilon\)\(\lambda\)ος,
\textbackslash{}., 2016. \(\Gamma\)ΕΝΕΤΙΚΗ \(\Pi\)ΟΙΚΙ\(\Lambda\)ΟΤΗΤΑ
\(\Sigma\)\(\Upsilon\)Μ\(\Pi\)ΑΤΡΙΚ\(\Omega\)Ν
\(\Pi\)\(\Lambda\)Η\(\Theta\)\(\Upsilon\)\(\Sigma\)Μ\(\Omega\)Ν
ΚΑ\(\Sigma\)ΤΑΝΙΑ\(\Sigma\) (CASTANEA SATIVA MILL.) ΚΑΙ Τ\(\Omega\)Ν
ΚΑΡ\(\Pi\)Ο\(\Phi\)Α\(\Gamma\)\(\Omega\)Ν ΕΝΤΟΜ\(\Omega\)Ν
Α\(\Upsilon\)ΤΗ\(\Sigma\)
(M\(\epsilon\)\(\tau\)\(\alpha\)\(\pi\)\(\tau\)\(\upsilon\)\(\chi\)\(\iota\)\(\alpha\)\(\kappa\)ή
\(\Delta\)\(\iota\)\(\alpha\)\(\tau\)\(\rho\)\(\iota\)\(\beta\)ή).
Α\(\rho\)\(\iota\)\(\sigma\)\(\tau\)ο\(\tau\)έ\(\lambda\)\(\epsilon\)\(\iota\)ο
\(\Pi\)\(\alpha\)\(\nu\)\(\epsilon\)\(\pi\)\(\iota\)\(\sigma\)\(\tau\)ή\(\mu\)\(\iota\)ο
\(\Theta\)\(\epsilon\)\(\sigma\)\(\sigma\)\(\alpha\)\(\lambda\)ο\(\nu\)ί\(\kappa\)\(\eta\)ς.


\end{document}
