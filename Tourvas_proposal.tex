\documentclass[12pt,a4paper,]{report}
\usepackage{lmodern}
\usepackage{setspace}
\setstretch{1.15}
\usepackage{amssymb,amsmath}
\usepackage{ifxetex,ifluatex}
\usepackage{fixltx2e} % provides \textsubscript
\ifnum 0\ifxetex 1\fi\ifluatex 1\fi=0 % if pdftex
  \usepackage[T1]{fontenc}
  \usepackage[utf8]{inputenc}
\else % if luatex or xelatex
  \ifxetex
    \usepackage{mathspec}
  \else
    \usepackage{fontspec}
  \fi
  \defaultfontfeatures{Ligatures=TeX,Scale=MatchLowercase}
    \setmainfont[]{Liberation Serif}
\fi
% use upquote if available, for straight quotes in verbatim environments
\IfFileExists{upquote.sty}{\usepackage{upquote}}{}
% use microtype if available
\IfFileExists{microtype.sty}{%
\usepackage{microtype}
\UseMicrotypeSet[protrusion]{basicmath} % disable protrusion for tt fonts
}{}
\usepackage[margin=1in]{geometry}
\usepackage{hyperref}
\hypersetup{unicode=true,
            pdftitle={Ανάλυση γενετικής ποικιλότητας και γονιδιακής ροής καστανεώνα και απομονωμένου φυσικού πληθυσμού καστανιάς στη νήσο Λέσβο με ουδέτερους και μη μοριακούς δείκτες},
            pdfauthor={Νικόλαος Τουρβάς},
            pdfborder={0 0 0},
            breaklinks=true}
\urlstyle{same}  % don't use monospace font for urls
\usepackage{longtable,booktabs}
\usepackage{graphicx,grffile}
\makeatletter
\def\maxwidth{\ifdim\Gin@nat@width>\linewidth\linewidth\else\Gin@nat@width\fi}
\def\maxheight{\ifdim\Gin@nat@height>\textheight\textheight\else\Gin@nat@height\fi}
\makeatother
% Scale images if necessary, so that they will not overflow the page
% margins by default, and it is still possible to overwrite the defaults
% using explicit options in \includegraphics[width, height, ...]{}
\setkeys{Gin}{width=\maxwidth,height=\maxheight,keepaspectratio}
\IfFileExists{parskip.sty}{%
\usepackage{parskip}
}{% else
\setlength{\parindent}{0pt}
\setlength{\parskip}{6pt plus 2pt minus 1pt}
}
\setlength{\emergencystretch}{3em}  % prevent overfull lines
\providecommand{\tightlist}{%
  \setlength{\itemsep}{0pt}\setlength{\parskip}{0pt}}
\setcounter{secnumdepth}{0}
% Redefines (sub)paragraphs to behave more like sections
\ifx\paragraph\undefined\else
\let\oldparagraph\paragraph
\renewcommand{\paragraph}[1]{\oldparagraph{#1}\mbox{}}
\fi
\ifx\subparagraph\undefined\else
\let\oldsubparagraph\subparagraph
\renewcommand{\subparagraph}[1]{\oldsubparagraph{#1}\mbox{}}
\fi

%%% Use protect on footnotes to avoid problems with footnotes in titles
\let\rmarkdownfootnote\footnote%
\def\footnote{\protect\rmarkdownfootnote}

%%% Change title format to be more compact
\usepackage{titling}

% Create subtitle command for use in maketitle
\newcommand{\subtitle}[1]{
  \posttitle{
    \begin{center}\large#1\end{center}
    }
}

\setlength{\droptitle}{-2em}

  \title{Ανάλυση γενετικής ποικιλότητας και γονιδιακής ροής καστανεώνα και
απομονωμένου φυσικού πληθυσμού καστανιάς στη νήσο Λέσβο με ουδέτερους
και μη μοριακούς δείκτες}
    \pretitle{\vspace{\droptitle}\centering\huge}
  \posttitle{\par}
  \subtitle{~}
  \author{Νικόλαος Τουρβάς}
    \preauthor{\centering\large\emph}
  \postauthor{\par}
      \predate{\centering\large\emph}
  \postdate{\par}
    \date{Οκτώβριος 2018}

\usepackage[greek]{babel}

\begin{document}
\maketitle

\section{Εισαγωγή}

Η ευρωπαϊκή καστανιά (\emph{Castanea sativa} Mill.) είναι ένα από τα πιο
διαδεδομένα και οικονομικά σημαντικά είδη της λεκάνης της Μεσογείου
(cite euforgen). Εκμεταλλεύεται τόσο για τους καρπούς της όσο και για
την ξυλεία της και για αυτό το λόγο καλλιεργείται σε περιοχές πλέον της
μεσογειακής ζώνης (citation?).

Η μεγάλη εξάρτηση κοινωνιών έχει οδηγήσει\ldots{} Εξαιτίας της
εκτεταμένης ανθρωπογενούς μεταφοράς γενετικού υλικού\ldots{}, γενετικό
υλικό του είδους έχει τύχει εκτεταμένης μεταφοράς (δες στα παπερ για
ancient και κάνε cite) εδώ και αιώνες. Επιπλέον η χρήση επιλεγμένων
γενοτύπων σε καστανεώνες αλλά και πρεμνοφυής διαχείριση των φυσικών
πληθυσμών έχουν επιφέρει σημαντικές απώλειες γενετικών πόρων του είδους
(αραβανοπ? - παπαδ). Η πίεση αυτή έχει ισχυροποιηθεί - ενταθεί λόγω
προσβολών κυρίως από τους μύκητες \emph{Cryphonectria parasitica} και
\emph{Phytophthora cambivora}. + κατακερματισμός απομόνωση γενετικού
υλικού

ΔΕΣ Παπαδήμα Καστοριά

\hypertarget{-}{%
\section{Θεωρητικό πλαίσιο}\label{-}}

\begin{enumerate}
\def\labelenumi{\arabic{enumi}.}
\item
  Σημασία γενετικής ποικ Long term see introduction (Pollegioni et al.,
  2011) γανο σελ 20-22 Selection may erode genetic diversity, which is a
  crucial factor for the success of breeding programs. Γεωργία vs
  δασοπονία φυσικοί πληθυσμοί και Γανό σελ 19 (Γανόπουλος, 2013) τι
  γίνεται με τους καστανεώνες, δες Πετρόπουλος. (Πετρόπουλος, 2016)

  τι γινεται με τα coppice --\textgreater{} (Aravanopoulos, 2018)
\end{enumerate}

Refugia - Gene flow - Long term (Petit et al., 2003, 2005) Founder
effect από ανθρω

(κίνδυνος για γονιδιακή ροή - μόλυνση στον φυσικό πληθυσμό)

Δυο λόγια για τη διεθνή βιβλιογραφία. Υπάρχει ή δεν υπάρχει ποικιλότητα;
"Γενικα'έχει βρεθεί ότι \ldots{} {[}\ldots{}{]} Συγκεκριμένα στην
Ελλάδα\ldots{}

Η λεπτομερής γνώση της δομής και της σύνθεσης της γενετικής
παραλλακτικότητας αποτελεί απαραίτητη προϋπόθεση για την προστασία και
διαχείριση των γενετικών πόρων ενός είδους. Στη γεωργία, οι περισσότερες
καλλιέργειες, έχουν απολέσει το μεγαλύτερο μέρος των γενετικών πόρων που
διέθεταν πριν την βελτιστοποίηση της παραγωγικής διαδικασίας μέσω της
επιλογής ποικιλιών και άριστων κλώνων. Παρόμοια τάση παρατηρείται και
στους καστανεώνες της χώρας μας οι οποίοι συνήθως δημιουργούνται με
εμβόλια επιλεγμένων κλώνων σε υποκείμενα άγριου τύπου. Αυτό έχει σαν
αποτέλεσμα να δυσχερένεται το έργο των βελτιωτών στην εύρεση κατάλληλων
γενοτύπων για διασταυρώσεις και να μειώνεται το δυναμικό προσαρμογής ως
προς νέα γνωρίσματα.

coppice - πρεμνοφυής διαχείριση - τα πρεμνοφυή δάση αναπαράγονται
βλαστικά και αρα αναμένεται να έχουν χαμηλότερο δραστικό μέγεθος
πληθυσμού συγκρινόμενα με όμοιας έκτασης φυσικά δάση καστανιάς που
αναπαράγονται εγγενώς (Mattioni et al., 2008).{[}hill{]} Το χαμηλότερο
δραστικό μέγεθος πληθυσμού υποδεικνύει ότι το γενετικό απόθεμα από το
οποίο προκύπτει η εξέλιξη του δάσους είναι μικρό και άρα ενδυνάμει
μικρότερο δυναμικό προσαρμογής ή/και πιο αργό βαθμό εξελικτικής
προσαρμογής σε περιπτώσεις ταχείας περιβαλλοντικής μεταβολής {[}47{]}.

refugia - founder effects

γονιδιακή ροή {[}aravanop2018{]}

\hypertarget{-}{%
\section{Ερευνητικά ερωτήματα}\label{-}}

Προτείνεται η μελέτη δύο πληθυσμών καστανιάς της νήσου Λέσβου. Ο ένας
πληθυσμός εντοπίζεται σε φυσική συστάδα η οποία διαχειρίζεται πρεμνοφυώς
(βλαστικός πολλαπλασιασμός), ενώ ο δεύτερος αντιπροσωπεύει καστανεώνα
(περιοχή Αγιάσου) ο οποίος έχει ιδρυθεί με τη χρήση εμβολίων σε
άγριου-τύπου υποκείμενα.

Σκοποί της παρούσας πρότασης είναι:

\begin{enumerate}
\def\labelenumi{\arabic{enumi}.}
\item
  Η μελέτη της γενετικής ποικιλότητας και διαφοροποίησης δύο πληθυσμών
  καστανιάς (καστανεώνας - φυσικός πληθυσμός) στη νήσο Λέσβο με τη χρήση
  μοριακών δεικτών μικροδορυφόρων SSR.
\item
  Η καταγραφή των κλώνων (multilocus genotypes) που συγκροτούν τον
  καστανεώνα και κατ' επέκταση την ποικιλία καστάνων Αγιάσου στη νήσο
  Λέσβο.
\item
  Η μελέτη γονιδιακής ροής μεταξύ των δύο πληθυσμών.
\end{enumerate}

\section{Μεθοδολογία}

\hypertarget{----dna}{%
\subsubsection{\texorpdfstring{\emph{Συλλογή φυτικού υλικού \& Εκχύλιση
DNA}}{Συλλογή φυτικού υλικού \& Εκχύλιση DNA}}\label{----dna}}

Θα διεξαχθεί δειγματοληψία κατ' ελάχιστον 25-30 ατόμων ανά πληθυσμό. Η
εκχύλιση του γενωμικού DNA θα πραγματοποιηθεί με τη μέθοδο CTAB (Doyle
and Doyle, 1987).

\hypertarget{---}{%
\subsection{Αλυσιδωτή αντίδραση πολυμεράσης \& Γενοτύπηση}\label{---}}

Για τη γενετική ανάλυση θα χρησιμοποιηθούν μοριακοί δείκτες SSR (Simple
Sequence Repeats). Οι SSR ή μικροδορυφορικοί δείκτες αποτελούνται από
αλληλουχίες 2-5 νουκλεοτιδίων που επαναλαμβάνονται 5-100 φορές (Budak et
al., 2004). Ο συγκυρίαρχος χαρακτήρας τους σε συνδυασμό με τα υψηλά
επίπεδα πολυμορφισμού που παρουσιάζουν, τους καθιστούν εξαιρετικούς
δείκτες για τη μελέτη πρόσφατων μικροεξελικτικών διεργασιών (Wang,
2010). Οι μικροδορυφόροι μπορούν να διακριθούν σε ουδέτερους και
λειτουργικούς. Στην πρώτη κατηγορία ανήκουν εκείνες οι γονιδιακές θέσεις
οι οποίες βρίσκονται σε τυχαία τμήματα του γονιδιώματος. Αντίθετα οι
λειτουργικοί δείκτες ενισχύουν γονιδιακές θέσεις που εκφράζονται και
επομένως σχετίζονται με κάποια βιολογική διεργασία {[}citation???{]}.

Συγκεκριμένα θα χρησιμοποιηθούν τρεις ουδέτεροι δείκτες SSR (EMCs38,
CsCAT3, CsCAT6) (Buck et al., 2003; Marinoni et al., 2003) και πέντε
δείκτες EST-SSR (GOT004, GOT021, FIR110, POR042 και WAG004) (Durand et
al., 2010). Οι EST-SSR που έχουν επιλεγεί σχετίζονται με την αντοχή των
δέντρων σε συνθήκες υδατικής έλλειψης.

\hypertarget{---}{%
\subsection{Γενετική ποικιλότητα εντός πληθυσμών}\label{---}}

Η στατιστική ανάλυση θα πραγματοποιηθεί σε περιβάλλον της στατιστικής
γλώσσας R (R Core Team, 2018) με χρήση των πακέτων poppr (Kamvar et al.,
2014), hierfstat (Goudet, 2005), και genepop (Rousset, 2008). Οι βασικές
παράμετροι γενετικής ποικιλότητας (αριθμός αλληλομόρφων, ετεροζυγωτία,
συντελεστής ομομειξίας, ισορροπία Hardy-Weinberg κ.α.) θα εκτιμηθούν
αφού πρώτα διερευνηθεί η ανισορροπία σύνδεσης μεταξύ των γονιδιακών
θέσεων και η πιθανότητα ύπαρξης μηδενικών αλληλομόρφων.

Τα δραστικά μεγέθη των πληθυσμών θα εκτιμηθούν με το πρόγραμμα
N\textsubscript{e} Estimator (Do et al., 2014), ενώ η ύπαρξη γενετικών
στενωπών στην εξελικτική ιστορία των πληθυσμών θα διερευνηθεί με τα
προγράμματα BOTTLENECK (Piry et al., 1999) (πρόσφατο παρελθόν: 2
N\textsubscript{e} - 4 N\textsubscript{e} γενεές) και Arlequin
(Excoffier and Lischer, 2010) (M-ratio test, υποδεικνύει παλαιότερες
δημογραφικές μεταβολές).

Επίσης θα υπολογιστεί ο αριθμός των γενοτύπων (multilocus genotypes) που
απαντώνται στον υπό μελέτη καστανεώνα.

\hypertarget{---}{%
\subsection{Γενετική ποικιλότητα μεταξύ πληθυσμών}\label{---}}

Θα υπολογιστεί ο συντελεστής διαφοροποίησης F\textsubscript{ST} (Nei,
1987), η γενετική απόσταση κατά Nei (1978), καθώς και η γεωμετρική
απόσταση Cavalli-Sforza (Cavalli-Sforza and Edwards, 1967) με το πακέτο
hierfstat (Goudet, 2005). Οι πολυμεταβλητές αναλύσεις PCA, CA και η
ανάλυση μοριακής διακύμανσης AMOVA θα διεξαχθούν με το πακέτο adegenet
(Jombart, 2008). Επιπλέον το λογισμικό Structure (Pritchard et al.,
2000) θα χρησιμοποιηθεί για την περαιτέρω διερεύνηση της διαφοροποίησης
των δύο πληθυσμών.

\hypertarget{-}{%
\subsection{Γονιδιακή ροή}\label{-}}

Ο αριθμός των μεταναστών θα εκτιμηθεί αρχικά σύμφωνα με το κλασσικό
μοντέλο \(Nm = [(1 / F~ST~) - 1] / 4\) (Frankham et al., 2002). Ωστόσο,
αποτελεί παραδοχή της μεθόδου ότι η μετανάστευη είναι τάξεις μεγεθους
μεγαλύτερη σε σχέση με τις μεταλλάξεις. Λόγω του υψηλού ρυθμού
μεταλλάξεων που χαρακτηρίζει τους μικροδορυφόρους (περίπου
\(5 * 10^{-4}\)) (Whittaker et al., 2003) αυτή η παραδοχή ενδέχεται να
παραβιάζεται (Whitlock, 2011). Για το λόγο αυτό, ο αριθμός των
μεταναστών θα υπολογιστεί και με τη χρήση των εξειδικευμένων λογισμίκών
(coalescent-based models) Migrate (Beerli and Felsenstein, 2001) και
MIGRAINE (De Iorio et al., 2005).

\hypertarget{-}{%
\section{Αναμενόμενα αποτελέσματα}\label{-}}

καταγραφή γενετικής ποικιλότητας και δημογραφικών παραγόντων των δύο
πληθυσμών. Αυτό θα ενισχύσει την προσπάθεια προστασία των γενετικών
πόρων του είδους στο ΒΑ Αιγαίο και την ορθολογική διαχείριση του
πολλαπλασιαστικού υλικού --\textgreater{} πρέπει να γίνει κάτι για την
προστασία των γεν πόρων του φυσικού πληθυσμού;

γενετικός χαρακτηρισμός ποικιλία Λέσβου Αγιάσου --\textgreater{} για την
ταυτοποίηση και την ιχνηλασιμότητα πιστοποίηση {[}δες και γανό{]}

\section{Βιβλιογραφία}

\hypertarget{refs}{}
\leavevmode\hypertarget{ref-Aravanopoulos2018a}{}%
Aravanopoulos, F.A., 2018. Do Silviculture and Forest Management Affect
the Genetic Diversity and Structure of Long-Impacted Forest Tree
Populations? Forests 9, 14.

\leavevmode\hypertarget{ref-beerli_maximum_2001}{}%
Beerli, P., Felsenstein, J., 2001. Maximum likelihood estimation of a
migration matrix and effective population sizes in n subpopulations by
using a coalescent approach. PNAS 98, 4563--4568.
\url{https://doi.org/10.1073/pnas.081068098}

\leavevmode\hypertarget{ref-buck_isolation_2003}{}%
Buck, E.J., Hadonou, M., James, C.J., Blakesley, D., Russell, K., 2003.
Isolation and characterization of polymorphic microsatellites in
European chestnut (Castanea sativa Mill.). Molecular Ecology Notes 3,
239--241. \url{https://doi.org/10.1046/j.1471-8286.2003.00410.x}

\leavevmode\hypertarget{ref-budak_comparative_2004}{}%
Budak, H., Shearman, R.C., Parmaksiz, I., Dweikat, I., 2004. Comparative
analysis of seeded and vegetative biotype buffalograsses based on
phylogenetic relationship using ISSRs, SSRs, RAPDs, and SRAPs. Theor
Appl Genet 109, 280--288.
\url{https://doi.org/10.1007/s00122-004-1630-z}

\leavevmode\hypertarget{ref-cavalli1967phylogenetic}{}%
Cavalli-Sforza, L.L., Edwards, A.W., 1967. Phylogenetic analysis: Models
and estimation procedures. Evolution 21, 550--570.

\leavevmode\hypertarget{ref-de_iorio_stepwise_2005}{}%
De Iorio, M., Griffiths, R.C., Leblois, R., Rousset, F., 2005. Stepwise
mutation likelihood computation by sequential importance sampling in
subdivided population models. Theoretical Population Biology, John
maynard smith memorial issue 68, 41--53.
\url{https://doi.org/10.1016/j.tpb.2005.02.001}

\leavevmode\hypertarget{ref-Do2014}{}%
Do, C., Waples, R.S., Peel, D., Macbeth, G.M., Tillett, B.J., Ovenden,
J.R., 2014. NeEstimator v2: Re-implementation of software for the
estimation of contemporary effective population size (Ne) from genetic
data. Molecular Ecology Resources 14, 209--214.
\url{https://doi.org/10.1111/1755-0998.12157}

\leavevmode\hypertarget{ref-doyle1987genomic}{}%
Doyle, J., Doyle, J., 1987. Genomic plant DNA preparation from fresh
tissue-CTAB method. Phytochem Bull 19, 11--15.

\leavevmode\hypertarget{ref-durand_fast_2010}{}%
Durand, J., Bodénès, C., Chancerel, E., Frigerio, J.-M., Vendramin, G.,
Sebastiani, F., Buonamici, A., Gailing, O., Koelewijn, H.-P., Villani,
F., Mattioni, C., Cherubini, M., Goicoechea, P.G., Herrán, A., Ikaran,
Z., Cabané, C., Ueno, S., Alberto, F., Dumoulin, P.-Y., Guichoux, E., de
Daruvar, A., Kremer, A., Plomion, C., 2010. A fast and cost-effective
approach to develop and map EST-SSR markers: Oak as a case study. BMC
Genomics 11, 570. \url{https://doi.org/10.1186/1471-2164-11-570}

\leavevmode\hypertarget{ref-excoffier_arlequin_2010}{}%
Excoffier, L., Lischer, H.E.L., 2010. Arlequin suite ver 3.5: A new
series of programs to perform population genetics analyses under Linux
and Windows. Molecular Ecology Resources 10, 564--567.
\url{https://doi.org/10.1111/j.1755-0998.2010.02847.x}

\leavevmode\hypertarget{ref-frankham_introduction_2002}{}%
Frankham, R., Briscoe, D.A., Ballou, J.D., 2002. Introduction to
conservation genetics. Cambridge university press.

\leavevmode\hypertarget{ref-goudet_hierfstat_2005}{}%
Goudet, J., 2005. Hierfstat, a package for r to compute and test
hierarchical F-statistics. Molecular Ecology Notes 5, 184--186.
\url{https://doi.org/10.1111/j.1471-8286.2004.00828.x}

\leavevmode\hypertarget{ref-Jombart2008}{}%
Jombart, T., 2008. Adegenet: A R package for the multivariate analysis
of genetic markers. Bioinformatics 24, 1403--1405.
\url{https://doi.org/10.1093/bioinformatics/btn129}

\leavevmode\hypertarget{ref-Kamvar2014}{}%
Kamvar, Z.N., Tabima, J.F., Grünwald, N.J., 2014. Poppr : An R package
for genetic analysis of populations with clonal, partially clonal,
and/or sexual reproduction. PeerJ 2, e281.
\url{https://doi.org/10.7717/peerj.281}

\leavevmode\hypertarget{ref-marinoni_development_2003}{}%
Marinoni, D., Akkak, A., Bounous, G., Edwards, K.J., Botta, R., 2003.
Development and characterization of microsatellite markers in Castanea
sativa (Mill.). Molecular Breeding 11, 127--136.
\url{https://doi.org/10.1023/A:1022456013692}

\leavevmode\hypertarget{ref-Mattioni2008}{}%
Mattioni, C., Cherubini, M., Micheli, E., Villani, F., Bucci, G., 2008.
Role of domestication in shaping \emph{Castanea} \emph{Sativa} genetic
variation in Europe. Tree Genetics \& Genomes 4, 563--574.
\url{https://doi.org/10.1007/s11295-008-0132-6}

\leavevmode\hypertarget{ref-nei1987molecular}{}%
Nei, M., 1987. Molecular evolutionary genetics. Columbia university
press.

\leavevmode\hypertarget{ref-Nei1978}{}%
Nei, M., 1978. Estimation of average heterozygosity and genetic distance
from a small number of individuals. Genetics 89, 583--590.
\url{https://doi.org/10.3390/ijms15010277}

\leavevmode\hypertarget{ref-Petit2003}{}%
Petit, R.J., Aguinagalde, I., De Beaulieu, J.L., Bittkau, C., Brewer,
S., Cheddadi, R., Ennos, R., Fineschi, S., Grivet, D., Lascoux, M.,
Mohanty, A., Müller-Starck, G., Demesure-Musch, B., Palmé, A., Martín,
J.P., Rendell, S., Vendramin, G.G., 2003. Glacial refugia: Hotspots but
not melting pots of genetic diversity. Science 300, 1563--1565.
\url{https://doi.org/10.1126/science.1083264}

\leavevmode\hypertarget{ref-Petit2005}{}%
Petit, R.J., Duminil, J., Fineschi, S., Hampe, A., Salvini, D.,
Vendramin, G.G., 2005. Comparative organization of chloroplast,
mitochondrial and nuclear diversity in plant populations. Molecular
Ecology 14, 689--701.
\url{https://doi.org/10.1111/j.1365-294X.2004.02410.x}

\leavevmode\hypertarget{ref-Piry1999}{}%
Piry, S., Luikart, G., Cornuet, J.-M., 1999. BOTTLENECK: A Computer
Program for Detecting Recent Reductions in the Effective Population Size
Using Allele Frequency Data. The Journal of Heredity Ann Hum Genet 90,
253--259.

\leavevmode\hypertarget{ref-Pollegioni2011}{}%
Pollegioni, P., Woeste, K., Olimpieri, I., Marandola, D., Cannata, F.,
Emilia Malvolti, M., 2011. Long-term human impacts on genetic structure
of Italian walnut inferred by SSR markers. Tree Genetics \& Genomes 7,
707--723. \url{https://doi.org/10.1007/s11295-011-0368-4}

\leavevmode\hypertarget{ref-pritchard_inference_2000}{}%
Pritchard, J.K., Stephens, M., Donnelly, P., 2000. Inference of
Population Structure Using Multilocus Genotype Data. Genetics 155,
945--959.

\leavevmode\hypertarget{ref-RCoreTeam2018}{}%
R Core Team, 2018. R: A Language and Environment for Statistical
Computing.

\leavevmode\hypertarget{ref-Rousset2008}{}%
Rousset, F., 2008. Genepop'007: A complete re-implementation of the
genepop software for Windows and Linux. Molecular Ecology Resources 8,
103--106. \url{https://doi.org/10.1111/j.1471-8286.2007.01931.x}

\leavevmode\hypertarget{ref-Wang2010a}{}%
Wang, I.J., 2010. Recognizing the temporal distinctions between
landscape genetics and phylogeography. Molecular Ecology 19, 2605--2608.
\url{https://doi.org/10.1111/j.1365-294X.2010.04715.x}

\leavevmode\hypertarget{ref-whitlock_gst_2011}{}%
Whitlock, M.C., 2011. G'ST and D do not replace FST. Molecular Ecology
20, 1083--1091. \url{https://doi.org/10.1111/j.1365-294X.2010.04996.x}

\leavevmode\hypertarget{ref-whittaker_likelihood-based_2003}{}%
Whittaker, J.C., Harbord, R.M., Boxall, N., Mackay, I., Dawson, G.,
Sibly, R.M., 2003. Likelihood-Based Estimation of Microsatellite
Mutation Rates. Genetics 164, 781--787.

\leavevmode\hypertarget{ref-__2013}{}%
Γανόπουλος, Ι., 2013. Διερεύνηση γενετικής ποικιλότητας, ταυτοποίηση και
εφαρμογή λειτουργικών δεικτών στην κερασιά (\emph{Prunus} \emph{Avium}
L.) (Διδακτορική Διατριβή). Αριστοτέλειο Πανεπιστήμιο Θεσσαλονίκης.

\leavevmode\hypertarget{ref-2016a}{}%
Πετρόπουλος, Σ., 2016. Γενετική ποικιλότητα συμπατρικών πληθυσμών
καστανιας (\emph{Castanea Sativa} Mill.) και των καρποφάγων εντόμων
αυτής (Mεταπτυχιακή Διατριβή). Αριστοτέλειο Πανεπιστήμιο Θεσσαλονίκης.


\end{document}
